\documentclass[aspectratio=1610]{beamer}
\usepackage[utf8]{inputenc}
\usepackage{multicol}
\usepackage[czech]{babel}
\usepackage{amsmath}
\usepackage{csquotes}
\usepackage{bold-extra}
\usepackage{listings}
\usepackage{alltt,xcolor}

\title{Směrování, odsazení, pseudotřídy}
\date{WBA | 9., 10. hodina}
\author[Cajthaml]{Matěj Cajthaml}

\usetheme{material}

\usePrimaryCustom

\begin{document}

\begin{frame}
\titlepage
\end{frame}

\begin{frame}{Obsah prezentace}
    \begin{cardTiny}
        \begin{minipage}{\textwidth}
            \vspace{1ex}
            \tableofcontents
        \end{minipage}
    \end{cardTiny}
\end{frame}


\section{Opakování}

\begin{frame}{Opakování}
    \begin{cardTiny}
        \begin{center}
            \textbf{Co je to HTML entita?}
        \end{center}
    \end{cardTiny}
    \begin{cardTiny}
        \begin{center}
            \textbf{Naco slouží vlastnost letter-spacing?}
        \end{center}
    \end{cardTiny}
    \begin{cardTiny}
        \begin{center}
            \textbf{Jak nastavíme barvu textu a pozadí?}
        \end{center}
    \end{cardTiny}
    \begin{cardTiny}
        \begin{center}
            \textbf{Jaký je rozdíl mezi identifikátorem a třídou?}
        \end{center}
    \end{cardTiny}
    \begin{cardTiny}
        \begin{center}
            \textbf{Jaké znáte selektory?}
        \end{center}
    \end{cardTiny}
\end{frame}


\section{Směrování}

\begin{frame}{Odkazy}
    \begin{cardTiny}
        \begin{flushleft}
            Na stránce můžeme odkazovat na:

            - soubory HTML na dané stránce
            
            - další webové stránky
            
            - speciální odkazy (email, telefony, ...)
        \end{flushleft}
    \end{cardTiny}
\end{frame}

\begin{frame}{Odkazy}
    \begin{cardTiny}
        \begin{flushleft}
            Pomocí tagu \texttt{a} a atributu \texttt{href}.

            Může obsahovat další obsah.

            Funguje tzv. adresářový systém - složky.

            \vspace{2ex}

            Relativní vs. absolutní cesta.

            Rodičovská složka - \texttt{..}
        \end{flushleft}
    \end{cardTiny}
\end{frame}

\begin{frame}{E-maily}
    \begin{cardTiny}
        \begin{flushleft}
            \begin{alltt}\textbf{$<$\textcolor{red}{a} href=$"$mailto:matej.cajthaml@ssps.cz$"$$>$}Napište mi!\textbf{$<$/\textcolor{red}{a}$>$}\end{alltt}
        
            Otevření e-mailového klienta.
        \end{flushleft}
    \end{cardTiny}
\end{frame}

\begin{frame}{Telefon}
    \begin{cardTiny}
        \begin{flushleft}
            \begin{alltt}\textbf{$<$\textcolor{red}{a} href=$"$tel:+420000000000$"$$>$}Nevolejte mi!\textbf{$<$/\textcolor{red}{a}$>$}\end{alltt}
        
            Otevření aplikace na volání.
        \end{flushleft}
    \end{cardTiny}
\end{frame}

\begin{frame}{Odkaz na jinou stránku}
    \begin{cardTiny}
        \begin{flushleft}
            Zadáváme celou URL adresu. Všechny povinné části URL.

            \begin{alltt}\textbf{$<$\textcolor{red}{a} href=$"$https://ssps.cz/$"$$>$}Webové stránky Smíchovské SPŠ\textbf{$<$/\textcolor{red}{a}$>$}\end{alltt}
        \end{flushleft}
    \end{cardTiny}
\end{frame}

\begin{frame}{Odkaz na stejnou stránku}
    \begin{cardTiny}
        \begin{flushleft}
            Zadáváme absolutní nebo relativní cestu se souborem a příponou.

            \begin{alltt}\textbf{$<$\textcolor{red}{a} href=$"$studenti/cajthaml.html$"$$>$}Cajthaml\textbf{$<$/\textcolor{red}{a}$>$}\end{alltt}
        \end{flushleft}
    \end{cardTiny}
\end{frame}


\section{Návěstí}

\begin{frame}{Návěstí}
    \begin{cardTiny}
        \begin{flushleft}
            = \textbf{jedinečný záchytný bod webové stránky}.

            Nekonečné množství.

            V URL se nazývá kotva.

            Určuje se pomocí atributu \texttt{id} - identifikátoru.

            Kotvě - navěstí se říká až při použítí \texttt{\#<nazev>}.

            \begin{alltt}\textbf{$<$\textcolor{red}{a} href=$"$\#kontakt$"$$>$}Vzhůru na kontakt\textbf{$<$/\textcolor{red}{a}$>$}\end{alltt}
        \end{flushleft}
    \end{cardTiny}
\end{frame}

\begin{frame}{Návěstí}
    \begin{cardTiny}
        \begin{flushleft}
            Absence tagu s identifikátorem.
            
            \vspace{2ex}

            Při kliku se stránka naroluje na pozici tagu.

            \vspace{2ex}

            Chování na začátku a na konci stránky.
        \end{flushleft}
    \end{cardTiny}
\end{frame}


\section{Nové tagy}

\begin{frame}{Blokový prvek - div}
    \begin{cardTiny}
        \begin{flushleft}
            Podobný textovému prvku span. Nemá žádný styl. Určen pro seskupení informací a stylování.
            
            \vspace{2ex}

            Zápis: \begin{alltt}\textbf{$<$\textcolor{red}{div}$>$}...\textbf{$<$/\textcolor{red}{div}$>$}\end{alltt}
        \end{flushleft}
    \end{cardTiny}
\end{frame}

\begin{frame}{Linkování stylů - link}
    \begin{cardTiny}
        \begin{flushleft}
            Již použit při základech linkování stylů. Potřebné atributy \texttt{rel} a \texttt{href}.
            
            Nepárový tag.

            \vspace{2ex}

            Zápis: \begin{alltt}\textbf{$<$\textcolor{red}{link} rel=$"$stylesheet$"$ href=$"$style.css$"$$>$}\end{alltt}
        \end{flushleft}
    \end{cardTiny}
\end{frame}

\begin{frame}{Naformátovaný text - pre}
    \begin{cardTiny}
        \begin{flushleft}
            Podobný obyčejnému paragrafu. Neupravuje zadaný text, zanechává tabulátory, mezery a zalomení řádků.

            Monospaced font.

            \vspace{2ex}

            Zápis: \begin{alltt}\textbf{$<$\textcolor{red}{pre}$>$}...\textbf{$<$/\textcolor{red}{pre}$>$}\end{alltt}
        \end{flushleft}
    \end{cardTiny}
\end{frame}


\section{Odsazení}

\begin{frame}
    \begin{center}
        \includegraphics[width=\textwidth]{img/sizing-chromium.png}
    \end{center}
\end{frame}

\begin{frame}{Odsazení}
    \begin{cardTiny}
        \begin{flushleft}
            Velmi důležitá část webových stránek.

            \vspace{2ex}

            Každý tag má předefinované styly.
            
            \vspace{2ex}
        
            Přepisování stylů.
        \end{flushleft}
    \end{cardTiny}
\end{frame}

\begin{frame}{Velikost}
    \begin{cardTiny}
        \begin{flushleft}
            Vlastnosti \texttt{width} a \texttt{height}.

            \vspace{2ex}

            Blokové elementy. 

            Nejčastěji px, cm, procenta, ...
        \end{flushleft}
    \end{cardTiny}
\end{frame}

\begin{frame}{Odsazení z venku}
    \begin{cardTiny}
        \begin{flushleft}
            Vlastnost \texttt{margin}.

            \vspace{2ex}

            Nepočitá se do velikosti tagu.

            Možnost určení pro každou stranu separé.

            Sdružená vlastnost.
        \end{flushleft}
    \end{cardTiny}
\end{frame}

\begin{frame}{Odsazení uvnitř}
    \begin{cardTiny}
        \begin{flushleft}
            Vlastnost \texttt{padding}.

            \vspace{2ex}

            Možnost určení pro každou stranu separé.

            Sdružená vlastnost.
        \end{flushleft}
    \end{cardTiny}
\end{frame}

\begin{frame}{Ohraničení}
    \begin{cardTiny}
        \begin{flushleft}
            Vlastnost \texttt{border}.

            \vspace{2ex}

            Možno nastavit pro každou stranu separé.

            Vlastnosti \texttt{width}, \texttt{style}, \texttt{color}.

            Sdružená vlastnost.
        \end{flushleft}
    \end{cardTiny}
\end{frame}

\begin{frame}{Ohraničení - zaoblení}
    \begin{cardTiny}
        \begin{flushleft}
            Vlastnost \texttt{border-radius}.

            \vspace{2ex}

            Nejčastěji px a \%.

            Možnost pro každý roh separé.

            Sdružená vlastnost.
        \end{flushleft}
    \end{cardTiny}
\end{frame}

\begin{frame}
    \begin{center}
        \includegraphics[width=\textwidth]{img/9-10-ukol-odsazeni.png}
    \end{center}
\end{frame}

\section{Nové vlastnosti}

\begin{frame}{text-transform}
    \begin{cardTiny}
        Nastavení velikost písmen v textu.

        Hodnoty: uppercase, lowercase, capitalize.

        \begin{alltt}
            \textcolor{red}{.special} \string{\\
                \textcolor{blue}{text-transform}: \textcolor{orange}{uppercase};\\
            \string}
        \end{alltt}
    \end{cardTiny}
\end{frame}

\begin{frame}{text-align-last}
    \begin{cardTiny}
        Podobné text-align, jen určuje poslední řádku.

        \begin{alltt}
            \textcolor{red}{.special} \string{\\
                \textcolor{blue}{text-align}: \textcolor{orange}{justify};\\
                \textcolor{blue}{text-align-last}: \textcolor{orange}{right};\\
            \string}
        \end{alltt}
    \end{cardTiny}
\end{frame}


\section{Pseudotřídy}

\begin{frame}{Pseudotřída}
    \begin{cardTiny}
        \begin{flushleft}
            = \textbf{speciální typ selektoru, reflektující určitý stav tagu}.

            Např. najetí myši, stav zobrazení odkazu, první písmeno, ...

            V budoucnu různé animace.

            Možnost změnit všechny vlastnosti, i ty, které nebyly nastaveny.

            \begin{alltt}
                \textcolor{red}{.special:pseudotrida} \string{\\
                    \textcolor{gray}{/* styly */}\\
                \string}
            \end{alltt}
        \end{flushleft}
    \end{cardTiny}
\end{frame}

\begin{frame}{Pseudotřída - hover}
    \begin{cardTiny}
        \begin{flushleft}
            = \textbf{při najetí myši}.

            \begin{alltt}
                \textcolor{red}{.special:hover} \string{\\
                    \textcolor{gray}{/* styly */}\\
                \string}
            \end{alltt}
        \end{flushleft}
    \end{cardTiny}
\end{frame}

\begin{frame}{Pseudotřída - active}
    \begin{cardTiny}
        \begin{flushleft}
            = \textbf{aktivní prvek, např. zvýrazněný tabem}.

            \begin{alltt}
                \textcolor{red}{.special:active} \string{\\
                    \textcolor{gray}{/* styly */}\\
                \string}
            \end{alltt}
        \end{flushleft}
    \end{cardTiny}
\end{frame}

\begin{frame}{Pseudotřída - link}
    \begin{cardTiny}
        \begin{flushleft}
            = \textbf{nezobrazený odkaz}.

            \begin{alltt}
                \textcolor{red}{a:link} \string{\\
                    \textcolor{gray}{/* styly */}\\
                \string}
            \end{alltt}
        \end{flushleft}
    \end{cardTiny}
\end{frame}

\begin{frame}{Pseudotřída - visited}
    \begin{cardTiny}
        \begin{flushleft}
            = \textbf{zobrazený odkaz}.

            \begin{alltt}
                \textcolor{red}{a:visited} \string{\\
                    \textcolor{gray}{/* styly */}\\
                \string}
            \end{alltt}
        \end{flushleft}
    \end{cardTiny}
\end{frame}

\section{Shrnutí}

\begin{frame}{Shrnutí}
    \begin{cardTiny}
        \begin{center}
            \textbf{Co vše se dá nastavit ve vlastnosti border?}
        \end{center}
    \end{cardTiny}
    \begin{cardTiny}
        \begin{center}
            \textbf{Co je to sdružená vlastnost?}
        \end{center}
    \end{cardTiny}
    \begin{cardTiny}
        \begin{center}
            \textbf{K čemu slouží vlastnost text-align-last, word-spacing, padding?}
        \end{center}
    \end{cardTiny}
    \begin{cardTiny}
        \begin{center}
            \textbf{Co je to pseudotřída?}
        \end{center}
    \end{cardTiny}
    \begin{cardTiny}
        \begin{center}
            \textbf{Jakým tagem se určuje přeškrtnuté písmo?}
        \end{center}
    \end{cardTiny}
\end{frame}

\begin{frame}
    \begin{center}
        \includegraphics[width=\textwidth]{img/9-10-ukol-pseudotridy.png}
    \end{center}
\end{frame}

\begin{frame}{Domácí úkol}
    \begin{cardTiny}
        \begin{center}
            ÚKOL
        \end{center}
        \begin{flushleft}
            Vytvořte odstavec, který bude mít žluté pozadí, černý text, písmo Comic Sans MS ve variantě italics a tloušťce 900. Použijte vnitřní odsazení zleva a zezdola s hodnotou 20px. Udělejte zaoblené rohy vlevo dole a vpravo nahoře na 50\%. Nastavte mezery mezi písmeny na -2px. Velikost písma nastavte na 4 řádky normálního písma.
        \end{flushleft}
    \end{cardTiny}
\end{frame}

\end{document}